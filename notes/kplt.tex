\documentclass[10pt]{article}
\usepackage{amsmath,amsthm,amssymb,graphicx,titling,mathrsfs,tikz,tikz-cd}
\usepackage{mathtools}
\usepackage{mleftright}
\usepackage{tikz-cd}
\usepackage{todonotes}
\usepackage{color}
\usepackage{stmaryrd}
\usepackage{hyperref}
\usepackage{pgfplots}

\usepackage{listings}

\usepackage{xcolor}
\hypersetup{
    colorlinks,
    linkcolor={blue!80!black},
    citecolor={blue!80!black},
    urlcolor={blue!80!black}
}

\lstset{ %
  backgroundcolor=\color{white},   % choose the background color; you must add \usepackage{color} or \usepackage{xcolor}; should come as last argument
  basicstyle=\footnotesize\ttfamily,        % the size of the fonts that are used for the code
  breakatwhitespace=false,         % sets if automatic breaks should only happen at whitespace
  breaklines=true,                 % sets automatic line breaking
  captionpos=b,                    % sets the caption-position to bottom
  commentstyle=\color{mygreen},    % comment style
  deletekeywords={...},            % if you want to delete keywords from the given language
  escapeinside={\%*}{*)},          % if you want to add LaTeX within your code
  extendedchars=true,              % lets you use non-ASCII characters; for 8-bits encodings only, does not work with UTF-8
  frame=single,                    % adds a frame around the code
  keepspaces=true,                 % keeps spaces in text, useful for keeping indentation of code (possibly needs columns=flexible)
  keywordstyle=\color{blue},       % keyword style
  morekeywords={sage},            % if you want to add more keywords to the set
  rulecolor=\color{black},         % if not set, the frame-color may be changed on line-breaks within not-black text (e.g. comments (green here))
  showspaces=false,                % show spaces everywhere adding particular underscores; it overrides 'showstringspaces'
  showstringspaces=false,          % underline spaces within strings only
  showtabs=false,                  % show tabs within strings adding particular underscores
  stepnumber=2,                    % the step between two line-numbers. If it's 1, each line will be numbered
  tabsize=2,                     % sets default tabsize to 2 spaces
  title=\lstname                   % show the filename of files included with \lstinputlisting; also try caption instead of title
}

\usepackage[ruled,vlined]{algorithm2e}

\theoremstyle{plain}
\newtheorem{theorem}{Theorem}[subsection]
\newtheorem{lemma}[theorem]{Lemma}
\newtheorem{corollary}[theorem]{Corollary}
\newtheorem{proposition}[theorem]{Proposition}
\newtheorem{remark}[theorem]{Remark}
\newtheorem{definition}[theorem]{Definition}
\theoremstyle{definition}
\newtheorem{example}[theorem]{Example}
\newtheorem{exercise}[theorem]{Exercise}
\newtheorem{prob}[theorem]{Problem}

\newcommand{\iso}{\cong}
\newcommand{\op}{\operatorname}
\newcommand{\normlin}{\trianglelefteq}
\newcommand{\ideal}{\trianglelefteq}
\newcommand{\N}{\mathbb{N}}
\newcommand{\Z}{\mathbb{Z}}
\newcommand{\Q}{\mathbb{Q}}
\newcommand{\R}{\mathbb{R}}
\newcommand{\C}{\mathbb{C}}
\newcommand{\A}{\mathbb{A}}
\newcommand{\F}{\mathbb{F}}
\newcommand{\HH}{\mathbb{H}}
\newcommand{\p}{\mathfrak{p}}

\newcommand{\cat}[1]{{\normalfont\textbf{#1}}}
\newcommand{\Set}{\cat{Set}}
\newcommand{\Ring}{\cat{Ring}}
\newcommand{\Top}{\cat{Top}}
\newcommand{\Vect}{\cat{Vect}}

\newcommand{\Hom}{\op{Hom}}
\newcommand{\Obj}{\op{Obj}}
\newcommand{\End}{\op{End}}
\newcommand{\Aut}{\op{Aut}}
\newcommand{\nrd}{\op{nrd}}
\newcommand{\trd}{\op{trd}}
\newcommand{\disc}{\op{disc}}
\newcommand{\discrd}{\op{discrd}}
\newcommand{\chr}{\op{char}}
\newcommand{\Frac}{\op{Frac}}
\newcommand{\Pl}{\op{Pl}}
\newcommand{\Ram}{\op{Ram}}
\newcommand{\Cls}{\op{Cls}}
\newcommand{\Gen}{\op{Gen}}
\newcommand{\Typ}{\op{Typ}}
\newcommand{\rk}{\op{rk}}

\newcommand{\defeq}{\vcentcolon=}

\setlength{\droptitle}{-7em}
\title{KPLT paper}
\author{Joel Laity}

\begin{document}
\maketitle
\tableofcontents

\section{The algorithm in a special case}
\subsection{Description of the special case}

Theorem 9 of the paper says that if you can solve the ideal isogeny problem for a single maximal order then you can solve it for any maximal order.
We only describe the algorithm for a quaternion algebra with discriminant \( p \equiv 3 \pmod{4} \). The case \( p \equiv 1 \pmod{4} \) is similar, but more annoying.

Any quaternion algebra \( B \) with discriminant \( p \equiv 3 \pmod{4} \) is isomorphic to \( (-1, -p \, \mid \, \Q) \), that is \( B = \op{span}_\Q\{1, i, j, k\} \) with
\[
    i^2 = -1 \text{, } j^2 = -p \text{ and } ij = k = -ji .
\]

Since we can choose to solve the isogeny problem for any maximal order we will assume that the order is
\[
    O = \Z \langle i, (1+j) / 2 \rangle = \op{span}_\Z \{ (1+j) / 2, (i+k) / 2 , j, k \}.
\]
Note that \( \{ (1+j) / 2, (i+k) / 2 , j, k \} \) is a \( \Z \)-basis for \( O \) in Hermite normal form.

The precise statement of the problem we wish to solve is given below.

\begin{prob} \label{prob: general}
    Let \( p \equiv 3 \pmod{4} \).
    Let \( B =  (-1, -p \, \mid \, \Q) \) be a quaternion algebra.
    Let \( O = \Z \langle i, (1+j) / 2 \rangle \).
    Let \( R = \Z[i] \).
    Given a basis for a left \( O \)-ideal \( I \subset O \) and a prime number \( \ell \), find a basis for another left \( O \)-ideal \( J \subset O \) such that \( I \) and \( J \) are in the same class and \( \nrd(J) \) is a power of \( \ell \).
\end{prob}

If \( p \equiv 1 \pmod{4} \) then \( B \), \( O \) and \( R \) will be different as described in Section 2.3 of the paper.


\subsection{High level overview}


We will solve Problem \ref{prob: general} like this:

\begin{description}
    \item[Step 1]
    Replace \( I \) with an ideal in the same class but with reduced norm \( N \) where \( N \neq \ell \) is a large prime coprime to \( |\disc R| \) and \( p \) and such that \( \ell \) is a quadratic nonresidue modulo \( N \).

    \item[Step 2]
    Find \( \alpha \in I \) such that \( I = ON + O\alpha \).

    \item[Step 3]
    Find \( \gamma \in O \) with
    \[
        \nrd(\gamma) = N \ell^{e_0}
    \]
    for some \( e_0 \in \N \).

    \item[Step 4]
          {\color{red} This is wrong fix it!}
    Find \( \mu_0 \in Rj = \Z j + \Z k \subset  O \) such that
    \[
        \gamma\mu_0 \equiv \alpha \mod NO
    \]
    and \( \nrd(\mu_0) \not\equiv 0 \pmod{N} \).

    \item[Step 5]
    Find \( \mu \in  O \) and \( \lambda \in \Z \) such that
    \[
        \mu \equiv \lambda \mu_0 \mod NO
    \]
    and
    \[
        \nrd(\mu) = \ell^{e_1}.
    \]

    \item[Step 6]
    Set \( \beta = \gamma \mu \).
    Then
    \[
        J = I(\overline{\beta}/N)
    \]
    is an ideal in the same class with \( J \subset O \) and \( \nrd(J) \) is a power of \( \ell \).
\end{description}


We first show that the output is correct in \S 0.3 , we will then show how to complete steps 1-5 in sections \S 0.4-0.8.

\subsection{Proof the output is correct}
We will use the following lemma.
\begin{lemma}[Finding an equivalent ideal] \label{lem: change norm}
    Let \( p \equiv 3 \pmod{4} \).
    Let \( B =  (-1, -p \, \mid \, \Q) \) be a quaternion algebra.
    Let \( O = \Z \langle i, (1+j) / 2 \rangle \).
    Let \( I \) be a left \( O \)-ideal of reduced norm \( N \).
    Let \( \alpha \in I \).
    Let \( \gamma = \overline{\alpha} / N \).
    Then \( I \gamma \) is a left \( O \)-ideal of norm \( \nrd(\alpha) / N \) and \( I \gamma \subset O \).
\end{lemma}
\begin{proof}
    First observe that
    \[
        \nrd(I \gamma) = \nrd(I)\nrd(\gamma) = N \nrd(\alpha) / N^2 = \nrd(\alpha) / N.
    \]
    It remains to show that \( I\gamma \subset O \).
    We have
    \[
        I\gamma = I\overline{\alpha} / N \subset I\overline{I} \frac{1}{N} = NO \frac{1}{N} = O.
    \]
\end{proof}

We can now prove correctness.
Let \( \beta \) be as in Step 6.
Then
\[
    \beta = \gamma \mu = \lambda\gamma\mu_0 \equiv \lambda \alpha \pmod{NO}.
\]
Since \( I = ON + O \alpha \) we know that \( \beta \in I \).
We also know that
\[
    \nrd(\beta) = \nrd(\gamma) \nrd(\mu) = N\ell^{e_0 + e_1} .
\]
Now from Lemma \ref{lem: change norm} \( J = I\overline{\beta}/N \) has norm \( \nrd(\beta) / N = \ell^{e_0 + e_1} \) as required.

\subsection{Step 1}
For step 1 we need to solve the following problem.

\begin{prob}
    Let \( p \equiv 3 \pmod{4} \).
    Let \( B =  (-1, -p \, \mid \, \Q) \) be a quaternion algebra.
    Let \( O = \Z \langle i, (1+j) / 2 \rangle \).
    Let \( R = \Z[i] \).
    Given a basis for a left \( O \)-ideal \( I \subset O \) find a basis for another left \( O \)-ideal \( I' \subset O \) such that \( I \) and \( I' \) are in the same class and \( \nrd(I') \) is a large prime.
\end{prob}

Given a left \( O \)-ideal \( I \) we show how to compute another ideal \( J \subset O \) in the same class but with prime norm.
Generate a random element \( \alpha \in I \) by taking random linear combination of the basis elements where the coefficients are bounded by some bound \( m \).
Calculate the norm of \( \alpha \) and if \( \nrd(\alpha) = NM \) where \( M \) is prime then return \( I' = I \overline{\alpha} / N \).
By Lemma \ref{lem: change norm} \( I' \) has norm \( M \) else pick another random element and continue.

{\color{red} It is unclear to me how to analyse the running time of this.}


\subsection{Step 2}
For step 2 we need to solve the following problem.

\begin{prob} \label{prob: I = ON + Oalpha}
    Let \( p \equiv 3 \pmod{4} \).
    Let \( B =  (-1, -p \, \mid \, \Q) \) be a quaternion algebra.
    Let \( O = \Z \langle i, (1+j) / 2 \rangle \).
    Let \( R = \Z[i] \).
    Given a basis for a left \( O \)-ideal \( I \subset O \) with prime norm \( N \) find \( \alpha \in I \) such that \( I = ON + O\alpha \).
\end{prob}

\begin{lemma}[\( I = ON + O\alpha \)]
    Let \( B \) be quaternion algebra over \( \Q \).
    Let \( O \subset B \) be a maximal order.
    Let \( I \subsetneq O \) be a left \( O \)-ideal with prime norm \( N = \nrd(I) \).
    Let \( \alpha \in I \) and suppose \( \alpha \not\in NO \).
    Then
    \[
        I = ON + O \alpha .
    \]
\end{lemma}
\begin{proof}
    {\color{red} This is an exercise in Voights book. The following proof is probably not the one he had in mind when he wrote the exercise.}
    Since \( N = \nrd(I) \) we know that \( ON \subset I \).
    Thus from the correspondence theorem \( I / NO \) is an ideal in \( O / NO \).
    There is a algebra isomorphism \( O / NO \iso M_2(\Z / N\Z) \).
    We will actually prove this result in Lemma \ref{lem: O / NO isomorphism} but we can steal the result from the future for now.
    From Lemma \ref{lem: ideals of M2} we know that all the left ideals of \( M_2(\Z / N\Z) \) are principal and generated by any nonzero element.
    Thus \(  \alpha + NO \in I / NO \) generates \( I / NO \) since \( \alpha \not\in NO \).

    Let \( \beta \in I \).
    Then since \( I \) is principle \( \beta + NO = (\gamma_1 + NO)(\alpha + NO) = \gamma_1\alpha + NO\) for some \( \gamma_1 + NO \in O / NO \).
    Thus \( \beta = \gamma_1\alpha + \gamma_2N \) for some \( \gamma_2 \in O \).
    Thus \( I \subset NO + O\alpha \).
    Since the other inclusion is obvious we have \( I =  NO + O\alpha \).
\end{proof}

Thus all we need to do to solve Problem \ref{prob: I = ON + Oalpha} is choose any \( \alpha \in I \) such that \( \alpha \not \in NO \) and then \( I = ON + O \alpha \).

\subsection{Step 3}
For step 3 we need to solve the following problem.

\begin{prob}
    Let \( p \equiv 3 \pmod{4} \).
    Let \( B =  (-1, -p \, \mid \, \Q) \) be a quaternion algebra.
    Let \( O = \Z \langle i, (1+j) / 2 \rangle \).
    Let \( R = \Z[i] \).
    Let \( N \in \N \) be prime.
    Let \( \ell \neq N \) be prime.
    Find \( \gamma \in O \) with
    \[
        \nrd(\gamma) = N\ell^{e_0}
    \]
    for some \( e_0 \in \N \). {\color{red} Sufficiently large?}
\end{prob}

Given any integer \( M \) we show how to find \( \gamma \in O \) such that \( \nrd(\gamma) = M \). First choose \( (y, z) \in [-m , m]^2 \) at random until
\[
    r = M - pf(y, z) = M - py^2 - pz^2
\]
is prime.
Then use Cornaccia's algorithm to find \( t, x \in \Z \) such that \(  t^2 + x^2 = r.\)
Finally
\begin{align*}
    \nrd(t + xi + yj + zk)
      & = t^2 + x^2 + py^2 + pz^2       \\
      & = r + py^2 + pz^2               \\
      & = M - py^2 - pz^2 + py^2 + pz^2 \\
      & = M.
\end{align*}
Moreover \( t + xi + yj + zk \in O \) because \( t, x, y, z \in \Z \) and \( \Z \langle i ,j\rangle \subset O \).

{\color{red} It is unclear to me how to analyse the running time of this, or how to choose \( m \).}

\subsection{Step 4}
For step 4 we need to solve the following problem.

\begin{prob} \label{prob: gamma*mu_0 = alpha}
    Let \( p \equiv 3 \pmod{4} \).
    Let \( B =  (-1, -p \, \mid \, \Q) \) be a quaternion algebra.
    Let \( O = \Z \langle i, (1+j) / 2 \rangle \).
    Let \( R = \Z[i] \).
    Let \( N \in \N \) be prime.
    Let \( \alpha \in O \).
    Let \( \gamma \in O \).
    Find \( \mu_0 \in Rj \subset  O \) such that
    \[
        \gamma\mu_0 \equiv \alpha \mod NO
    \]
    and \( \nrd(\mu_0) \not\equiv 0 \pmod{N} \).
\end{prob}

The first observation is that instead of working over \( O / NO \) we can work over \( (R+Rj) / N(R+Rj) = \Z \langle i, j \rangle / N \Z \langle i, j \rangle\).

\begin{lemma}[\( R/ NR \iso S / NS \) ]
    Let \( R \) be a ring.
    Let \( S \subset R \) be a subring.
    Let \( M = [R : S] \) where the index is as abelian groups.
    Let \( N \in \N \) be coprime to \( M \).
    Let \( a, b \in \Z \) such that \( Ma + Nb = 1 \).
    Then the map
    \begin{align*}
          & S / NS \to R / NR         \\
          & s + NS \mapsto s + NR     \\
          & Mar + NS \mapsfrom r + NR
    \end{align*}
    is an isomorphism.
\end{lemma}
\begin{proof}
    From the second isomorphism theorem for rings the map
    \begin{align*}
          & \frac{S}{S \cap NR} \to \frac{S + NR}{NR} \\
          & s + S \cap NR \mapsto s + NR
    \end{align*}
    is an isomorphism.
    Thus it remains to show that \( S \cap NR = NS \) and \( S + NR = R \).

    Clearly \( NS \subset S \cap NR \).
    Let \( x \in S \cap NR \).
    Then \( x = Ny \) for \( y \in R \) and \( Ny \in S \).
    So \( Ny + S = 0 + S \) in \( R / S \), where the quotient is as abelian groups.
    Hence \( |y + S| \mid N \).
    Since \( |R / S| = M \) we also know that \( |y + S| \mid M \) so \( |y + S| \mid \gcd(N, M) = 1 \), which implies that \( y \in S \).
    Thus \( x = Ny \in NS \) so \( NS = S \cap NR \).

    For the second equality let \( r \in R \).
    Choose \( a, b \in \Z \) such that \( Na + Mb = 1 \).
    Then \( Nar + Mbr = r \).
    Now \( Nar \in NR \) and \( Mbr \in S \) since \( |r + S| \mid M \).
    Thus \( r \in S + NR \).
    Therefore \( S + NR = R \).
\end{proof}

\begin{corollary}[\( O / NO \iso (R + Rj) / N(R + Rj)\)]
    Let \( p \equiv 3 \pmod{4} \).
    Let \( B =  (-1, -p \, \mid \, \Q) \) be a quaternion algebra.
    Let \( O = \Z \langle i, (1+j) / 2 \rangle \).
    Let \( R = \Z[i] \).
    Let \( D =  [O : R + Rj] = 4\)
    Let \( N \) be a prime not dividing \( D \).
    Let \( a, b \in \Z \) such that \( Da + Nb = 1 \).
    Then
    \begin{align*}
          & \frac{R + Rj}{N(R + Rj)} \to O / NO \\
          & x + N(R + Rj)\mapsto x + NO         \\
          & Dax + N(R + Rj) \mapsfrom x + NO
    \end{align*}
    is an isomorphism
\end{corollary}

We care about constructing an isomorphism to \( (R + Rj) / N(R + Rj) \) because \( R = \Z[i] \) so
\[
    (R + Rj) / N(R + Rj) \iso (\Z + \Z i + \Z j + \Z k) / N (\Z + \Z i + \Z j + \Z k)
\]
and there is a natural isomoprhism
\[
    (\Z + \Z i + \Z j + \Z k) / N (\Z + \Z i + \Z j + \Z k) \iso (-1, -p \, |\, \Z / N \Z).
\]
Thus \( (R + Rj) / N(R + Rj) \) is a quaternion algebra over the finite field \( \Z / N\Z \).
Any quaternion algebra over a finite field is split, so we have proved the following lemma.

\begin{lemma} \label{lem: O / NO isomorphism}
    Let \( p \equiv 3 \pmod{4} \).
    Let \( B =  (-1, -p \, \mid \, \Z / N \Z) \) be a quaternion algebra.
    Let \( O = \Z \langle i, (1+j) / 2 \rangle \).
    Let \( R = \Z[i] \).
    Let \( D =  [O : R + Rj] = 4\)
    Let \( N \) be a prime not dividing \( D \).
    Then
    \[
        O / NO \iso (R + Rj) / N(R + Rj) \iso M_2(\Z / N \Z  ).
    \]
\end{lemma}


Thus to solve Problem \ref{prob: gamma*mu_0 = alpha} we can instead solve the same problem over the quaternion algebra \( ( -1, -p \, |\, \Z / N \Z) \).

\begin{prob}
    Let \( p \equiv 3 \pmod{4} \).
    Let \( N \neq p\) be prime.
    Let \( B =  (-1, -p \, \mid \, \Z / N \Z) \) be a quaternion algebra.
    Let \( \alpha, \gamma \in B \).
    Find \( \mu_0 \in \op{span}_{\Z / N / \Z}\{j, k\} \subset  B \) such that
    \[
        \gamma\mu_0 \equiv \alpha \mod NO
    \]
    and \( \nrd(\mu_0) \neq 0 \).
\end{prob}

This problem can be solved by finding the left action matrix of \( \gamma \) and solving a system of linear equations.

The code below demonstrates how you would do this.

\begin{example}
    -
    \begin{lstlisting}
        sage: F = GF(101)
        sage: B.<i, j, k> = QuaternionAlgebra(F, -1, -p)
        sage: gamma = 16 + 59*i
        sage: alpha = 10*j - k
        sage: alpha_vec = matrix(F, alpha.coefficient_tuple())
        sage: gamma_mat = gamma.matrix(action='left')
        sage: # Ensure our solution is in Rj
        sage: gamma_mat = matrix(F, [gamma_mat[2], gamma_mat[3]])
        sage: sol = gamma_mat.solve_left(alpha_vec)
        sage: y, z = [sol[0][i] for i in [0, 1]]
        sage: mu = y*j + z*k
        sage: gamma * mu == alpha
        True
    \end{lstlisting}
\end{example}

If no solution exists then repeat the previous step to find a different \( \gamma \).

\subsection{Step 5}
For step 5 we need to solve the following problem.

\begin{prob} \label{prob: mu = lambda * mu_0}
    Let \( p \equiv 3 \pmod{4} \).
    Let \( B =  (-1, -p \, \mid \, \Q) \) be a quaternion algebra.
    Let \( O = \Z \langle i, (1+j) / 2 \rangle \).
    Let \( R = \Z[i] \).
    Let \( N \) be prime.
    Given \( \mu_0 \in Rj \) with \( \nrd(\mu_0) \not\equiv 0 \pmod{N} \) find \( \mu \in O \) with
    \[
        \nrd(\mu) = \ell^{e_1}
    \]
    for some \( e_1 \in \N \) and \(\lambda \in \Z \) such that
    \[
        \mu \equiv \lambda \mu_0 \pmod{NO}.
    \]
\end{prob}

We first need a lemma.
\begin{lemma}[\( \nrd(\alpha_1 + \alpha_2) - \nrd(\alpha_1) - \nrd(\alpha_2) = \trd(\alpha_1 \overline{\alpha_2}) \)]
    Let \( B \) be a quaternion algebra.
    Let \( \alpha_1, \alpha_2 \in B \).
    Then
    \[
        \nrd(\alpha_1 + \alpha_2) - \nrd(\alpha_1) - \nrd(\alpha_2)
        = \trd(\alpha_1 \overline{\alpha_2})
    \]
\end{lemma}
\begin{proof}
    -
    \begin{lstlisting}
        sage: A.<t_1, x_1, y_1, z_1, t_2, x_2, y_2, z_2, a, b> = PolynomialRing(QQ)
        sage: B.<i, j, k> = QuaternionAlgebra(a, b)
        sage: w_1 = t_1 + x_1 * i + y_1 * j + z_1 * k
        sage: w_2 = t_2 + x_2 * i + y_2 * j + z_2 * k
        sage: (w_1 + w_2).reduced_norm() - w_1.reduced_norm() - w_2.reduced_norm() == (w_1 * w_2.conjugate()).reduced_trace()
        True
    \end{lstlisting}
\end{proof}


We will now show how to solve Problem \ref{prob: mu = lambda * mu_0}.
As input we are given a prime \( p \equiv 3 \pmod{4} \), \( \mu_0 = \beta_0j \in (R / NR)^*[j] \), and a prime \( N \neq p \) sufficiently large.
The goal is to find \( \lambda \in \Z \) and \( \mu_1 \in  R + Rj \) such that \( \mu = \lambda \mu_0 + N \mu_1 \) has \( \ell \) power norm
\[
    \nrd(\mu) = \ell^e.
\]

Write \( \mu_1 = \alpha_1 + \beta_1j \) where \( \alpha_1, \beta_1 \in \Z[i] \).
Then
\begin{align*}
    \ell^e
      & = \nrd(\mu)                                                                                                      \\
      & = \nrd(\lambda \mu_0 + N \mu_1)                                                                                  \\
      & = \nrd(N\alpha_1 + (\lambda\beta_0 + N\beta_1)j)                                                                 \\
      & = \nrd(N\alpha_1) +p\nrd(\lambda\beta_0 + N\beta_1)                                                              \\
      & = \nrd(N\alpha_1) + p\trd(\lambda \beta_0\overline{N\beta_1}) + p\nrd(\lambda\beta_0) + p\nrd(N\beta_1)          \\
      & = N^2\nrd(\alpha_1) +  p\lambda N \trd(\beta_0\overline{\beta_1}) + p\lambda^2 \nrd(\beta_0) + pN^2\nrd(\beta_1) \\
      & = p\lambda^2 \nrd(\beta_0) + Np\lambda \trd(\beta_0\overline{\beta_1}) + N^2(p\nrd(\beta_1) + \nrd(\alpha_1))
\end{align*}
Note that we are given all variables with subscript 0 and are solving for the variables with subscript 1.

The last line of the equation above is a polynomial in \( N \).
We can therefore solve the equation above by ``building up'' to the solution.
First we find a solution modulo \( N \), then modulo \( N^2 \), and finally we solve the entire equation.

Reducing modulo \( N \) gives
\[
    \ell^e \equiv p\lambda^2\nrd(\beta_0) \pmod{N}.
\]
We then choose \( e \) to be even if \( (p\nrd(\beta_0))^{-1} \bmod{N} \) is a square modulo \( N \) and odd otherwise.
This ensures that \( \ell^e(p\nrd(\beta_0))^{-1} \) is a square modulo \( N \) and we can solve for \( 0 <  \lambda < N\).
We also choose \( e \) sufficiently large (\( \ell^e = O(p|D|N^4) \)).

Reducing modulo \( N^2 \) gives
\[
    \frac{\ell^e - p\lambda^2\nrd(\beta_0)}{N}\equiv p\lambda \trd(\beta_0\overline{\beta_1}) \pmod{N^2}
\]
which is a linear equation in the coefficients of \( \beta_1 \in \Z[i]\).
There will be \( N \) solutions to this equation and we a solution randomly such that \( \nrd(\lambda\beta_0 + N\beta_1) < N^4 \).

The only variable left that hasn't be determined is \( \alpha_1 \).
We rearrange the equation \( \ell^e = \nrd(N\alpha_1) +p\nrd(\lambda\beta_0 + N\beta_1) \) and get
\[
    \nrd(\alpha_1)
    = \frac{\ell^e - p\nrd(\lambda\beta_0 + N\beta_1)}{N^2}.
\]
The integer \( e \) was chosen sufficiently large so that the right hand side is positive.

It is possible that no solution for \( \alpha_1 \) exists.
If this is the case then we choose a different \( \beta_1 \) from the previous step and repeat.
Under the assumption that the right hand side of the above equation behaves like random values around \( N^4|D|p \) as we change \( \beta_1 \), we will find a solution in polynomial time.

\appendix
\section{Matrices over finite fields}
\begin{lemma}[Ideals of \( M_2(\F_p) \)] \label{lem: ideals of M2}
    Let \( p \) be prime.
    Let \( R =  M_2(\F_p) \).
    Let \( a, b \in \F_p \) not both equal to zero.
    Then the set
    \[
        I = R
        \begin{pmatrix}
            a & b \\
            0 & 0
        \end{pmatrix}
        = \left\{
        \begin{pmatrix}
            xa & xb \\
            ya & yb
        \end{pmatrix}
        \mid x, y \in \F_p\right\}
        = \{\alpha \in R \mid \begin{pmatrix}-b & a \end{pmatrix}^T \in \ker \alpha \}
    \]
    is a nonzero, proper, left \( R \)-ideal.
    Moreover any nonzero, proper, left \( R \)-ideal is of this form.
    As a consequence we have
    \begin{enumerate}
        \item all left ideals of \( R \) are principal,
        \item for any proper ideal \( I \) any nonzero element in \( I \) will generate \( I \),
        \item if \( R\alpha = R\beta \) for \( \alpha, \beta \in R \) then \( \alpha \) is associate to \( \beta \).
    \end{enumerate}
\end{lemma}
\begin{proof}
    (Sketch).
    Suppose there were an ideal \( I \) not of this form.
    Choose any nonzero \( \alpha \in I \) and let \( \begin{pmatrix}a & b \end{pmatrix} \) be any nonzero row of \( \alpha \).
    By multiplying of the left with elements of \( R \) we can perfrom elementary row operations on \( \alpha \) while still keeping it in \( I \).
    Thus WLOG
    \[
        \alpha
        =
        \begin{pmatrix}
            a & b \\
            * & *
        \end{pmatrix} \in I.
    \]
    Suppose there is some \( \beta \in I \) which has a row \(\begin{pmatrix}c & d \end{pmatrix}  \not\in \op{span}_{\F_p}\{\begin{pmatrix}a & b \end{pmatrix} \} \).
    Then using the same trick we may assume WLOG that
    \[
        \beta
        =
        \begin{pmatrix}
            * & * \\
            c & d
        \end{pmatrix} \in I.
    \]
    Then
    \[
        \begin{pmatrix}
            a & b \\
            c & d
        \end{pmatrix}
        =
        \begin{pmatrix}
            1 & 0 \\
            0 & 0
        \end{pmatrix}
        \begin{pmatrix}
            a & b \\
            * & *
        \end{pmatrix}
        +
        \begin{pmatrix}
            0 & 0 \\
            0 & 1
        \end{pmatrix}
        \begin{pmatrix}
            * & * \\
            c & d
        \end{pmatrix}
        \in I
    \]
    but this matrix has rank 2 since \(\begin{pmatrix}c & d \end{pmatrix}  \not\in \op{span}_{\F_p}\{\begin{pmatrix}a & b \end{pmatrix} \} \) so it is invertible.
    Thus \( I = R \), a contradiction.

    Thhe consequences follow almost immediatly.
\end{proof}

\end{document}




































