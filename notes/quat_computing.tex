\documentclass[10pt]{article}
\usepackage{amsmath,amsthm,amssymb,graphicx,titling,mathrsfs,tikz,tikz-cd}
\usepackage{mathtools}
\usepackage{mleftright}
\usepackage{tikz-cd}
\usepackage{todonotes}
\usepackage{color}
\usepackage{stmaryrd}
\usepackage{hyperref}
\usepackage{pgfplots}

\usepackage{listings}

\usepackage{xcolor}
\hypersetup{
    colorlinks,
    linkcolor={blue!80!black},
    citecolor={blue!80!black},
    urlcolor={blue!80!black}
}

\lstset{ %
  backgroundcolor=\color{white},   % choose the background color; you must add \usepackage{color} or \usepackage{xcolor}; should come as last argument
  basicstyle=\footnotesize\ttfamily,        % the size of the fonts that are used for the code
  breakatwhitespace=false,         % sets if automatic breaks should only happen at whitespace
  breaklines=true,                 % sets automatic line breaking
  captionpos=b,                    % sets the caption-position to bottom
  commentstyle=\color{mygreen},    % comment style
  deletekeywords={...},            % if you want to delete keywords from the given language
  escapeinside={\%*}{*)},          % if you want to add LaTeX within your code
  extendedchars=true,              % lets you use non-ASCII characters; for 8-bits encodings only, does not work with UTF-8
  frame=single,                    % adds a frame around the code
  keepspaces=true,                 % keeps spaces in text, useful for keeping indentation of code (possibly needs columns=flexible)
  keywordstyle=\color{blue},       % keyword style
  morekeywords={sage},            % if you want to add more keywords to the set
  rulecolor=\color{black},         % if not set, the frame-color may be changed on line-breaks within not-black text (e.g. comments (green here))
  showspaces=false,                % show spaces everywhere adding particular underscores; it overrides 'showstringspaces'
  showstringspaces=false,          % underline spaces within strings only
  showtabs=false,                  % show tabs within strings adding particular underscores
  stepnumber=2,                    % the step between two line-numbers. If it's 1, each line will be numbered
  tabsize=2,                     % sets default tabsize to 2 spaces
  title=\lstname                   % show the filename of files included with \lstinputlisting; also try caption instead of title
}

\usepackage[ruled,vlined]{algorithm2e}

\theoremstyle{plain}
\newtheorem{theorem}{Theorem}[subsection]
\newtheorem{lemma}[theorem]{Lemma}
\newtheorem{corollary}[theorem]{Corollary}
\newtheorem{proposition}[theorem]{Proposition}
\newtheorem{remark}[theorem]{Remark}
\newtheorem{definition}[theorem]{Definition}
\theoremstyle{definition}
\newtheorem{example}[theorem]{Example}
\newtheorem{exercise}[theorem]{Exercise}
\newtheorem{problem}[theorem]{Problem}

\newcommand{\iso}{\cong}
\newcommand{\op}{\operatorname}
\newcommand{\normlin}{\trianglelefteq}
\newcommand{\ideal}{\trianglelefteq}
\newcommand{\N}{\mathbb{N}}
\newcommand{\Z}{\mathbb{Z}}
\newcommand{\Q}{\mathbb{Q}}
\newcommand{\R}{\mathbb{R}}
\newcommand{\C}{\mathbb{C}}
\newcommand{\A}{\mathbb{A}}
\newcommand{\F}{\mathbb{F}}
\newcommand{\HH}{\mathbb{H}}
\newcommand{\p}{\mathfrak{p}}

\newcommand{\cat}[1]{{\normalfont\textbf{#1}}}
\newcommand{\Set}{\cat{Set}}
\newcommand{\Ring}{\cat{Ring}}
\newcommand{\Top}{\cat{Top}}
\newcommand{\Vect}{\cat{Vect}}

\newcommand{\Hom}{\op{Hom}}
\newcommand{\Obj}{\op{Obj}}
\newcommand{\End}{\op{End}}
\newcommand{\Aut}{\op{Aut}}
\newcommand{\nrd}{\op{nrd}}
\newcommand{\trd}{\op{trd}}
\newcommand{\disc}{\op{disc}}
\newcommand{\discrd}{\op{discrd}}
\newcommand{\chr}{\op{char}}
\newcommand{\Frac}{\op{Frac}}
\newcommand{\Pl}{\op{Pl}}
\newcommand{\Ram}{\op{Ram}}
\newcommand{\Cls}{\op{Cls}}
\newcommand{\Gen}{\op{Gen}}
\newcommand{\Typ}{\op{Typ}}
\newcommand{\rk}{\op{rk}}

\newcommand{\defeq}{\vcentcolon=}

\setlength{\droptitle}{-7em}
\title{KPLT Paper}
\author{Joel Laity}

\begin{document}
\maketitle
\tableofcontents

\section{Properties in the special case}

\begin{lemma}[Properties of \( (-1, -p \, \mid \, \Q) \)]
    Let \( p \equiv 3 \pmod{4} \).
    Let \( B =  (-1, -p \, \mid \, \Q) \) be a quaternion algebra.
    Then the multiplication table for \( B \) is
    \[
        \begin{array}{c|ccc}
              & i  & j   & k  \\
            \hline
            i & -1 & k   & -j \\
            j & -k & -p  & pi \\
            k & j  & -pi & -p
        \end{array}
    \]

    The norm equation for \( B \) is
    \[
        \nrd(t + xi + yj + zk) = t^2 + x^2 + y^2p + z^2p.
    \]
    The trace equation for \( B \) is
    \[
        \trd(t + xi + yj + zk) = 2t.
    \]
\end{lemma}
\begin{proof}
    We verify with Sage.
    \begin{lstlisting}
        sage: A.<t, x, y, z, p> = PolynomialRing(QQ)
        sage: B.<i, j, k> = QuaternionAlgebra(-1, -p)
        sage: matrix([[a * b for b in [i, j, k]]
                       for a in [i, j, k]])
        [    -1      k     -j]
        [    -k     -p    p*i]
        [     j (-p)*i     -p]
        sage: alpha = t + x*i + y*j + z*k
        sage: alpha.reduced_norm()
        y^2*p + z^2*p + t^2 + x^2
        sage: alpha.reduced_trace()
        2*t
    \end{lstlisting}
\end{proof}

\begin{lemma}[]
    Let \( p \equiv 3 \pmod{4} \).
    Let \( B =  (-1, -p \, \mid \, \Q) \) be a quaternion algebra.
    Then \( \disc B = p \).
\end{lemma}
\begin{proof}
    (Sketch.)
    The discriminant of \( B \) is the product of the primes where \( B \) is ramified.
    Apparently?? \( B \) can only be ramified at primes dividing its invariants (\( a, b = -1, -p\)).
    Calculations with hilbert symbols then shows that \( \disc B = p \)
\end{proof}

\begin{lemma}[Maximal order in \( B_{p, \infty} \) where \( p \equiv 3 \pmod{4} \)]
    Let \( p \equiv 3 \pmod{4} \).
    Let \( B =  (-1, -p \, \mid \, \Q) \) be a quaternion algebra.
    Let \( O = \Z \langle i, (1+j) / 2 \rangle \).
    Then \( O \) is a maximal order.
\end{lemma}
\begin{proof}
    Check that \( \disc O  = p = \disc B\).
\end{proof}

\begin{example}
    Our running example will have \( p = 59 \).
    \begin{lstlisting}
        sage: p = 59
        sage: p.mod(4) == 3
        True 
        sage: B.<i, j, k> = QuaternionAlgebra(-1, -p)
        sage: B.discriminant() == p
        True
        sage: B.maximal_order()
        Order of Quaternion Algebra (-1, -59) with base ring Rational Field with basis (1/2 + 1/2*j, 1/2*i + 1/2*k, j, k)
        sage: # We now verify B.maximal_order() is maximal.
        sage: B.maximal_order().discriminant() == B.discriminant()
        True
    \end{lstlisting}
\end{example}

We will now define a ``distinguished quadratic subring'' of \( O \).
This subring may be slightly different if \( p \not\equiv 3 \pmod{4} \).

\begin{lemma}[Distinguished quadratice subring]
    Let \( p \equiv 3 \pmod{4} \).
    Let \( B =  (-1, -p \, \mid \, \Q) \) be a quaternion algebra.
    Let \( O = \Z \langle i, (1+j) / 2 \rangle \).
    Let \( R = \Z[i] \subset O \).
    Then \( R \) has discriminant \( D = 4 \), \( R^\perp = Rj \) and \( R + Rj = \Z\langle i, j \rangle\) is a suborder of \( O \) of index \( D \).
    Moreover
    \[
        \nrd(t + xi + (y + zi)j) = f(t, x) + p f(y, z)
    \]
    where \( f(t, x) = t^2 + x^2 \) is a principal quadratic form of discriminant \( D \).
\end{lemma}
\begin{proof}
    Note that it seems they are not normalizing their discriminants by \( 2^{-n} \) which is the usual thing to do.
    Hence the discriminant of \( f \) is 4 as required by the lemma.
    The discriminant is equal to the index since \( O \) is maximal so \( O = O_L(R) \).
\end{proof}


\section{Computing with quaternions}

\begin{definition}[Lattice]
    Let \( B \) be quaternion algebra over \( \Q \).
    A \emph{lattice} in \( B \) is a finitely generated \( \Z \)-submodule \( I \subset B \) with \( I\Q = B \).
\end{definition}

\begin{remark}
    The condition \( I\Q = B \) is equivalent ot saying that \( I \) contains a basis for B as a \( \Q \)-vector space.
\end{remark}

\begin{definition}[Lambda notation for lattices]
    Let \( A \) be a \( n \times k \) rational matrix.
    Define \( \Lambda(A) \subset \Q^n \) to be the \( \Z \)-submodule
    \[
        \Lambda(A) \vcentcolon = \{ Ax \mid x \in \Z^k \}.
    \]
\end{definition}

\begin{definition}[Representing a lattice as a matrix]
    Let \( B \) be quaternion algebra over \( \Q \).
    Let \( I \subset B \) be a lattice in \( B \).
    Let \( \alpha_1, \dots , \alpha_k \in I\) be generators for \( I \) as a \( \Z \)-module.
    Let
    \[
        A =
        \begin{pmatrix}
            [\alpha_1] \\  \vdots \\ [\alpha_k]
        \end{pmatrix}.
    \]
    where \( [\alpha] =  \begin{pmatrix} t, & x, & y, & z, \end{pmatrix} \) is the row of coefficents of \( \alpha = t + xi + yj + zk \).
    Then
    \[
        \Lambda(A) = I.
    \]
    We use \( A \) to represent the lattice \( I \) in a computer.
\end{definition}


\begin{example}
    The following code shows how to take a list of generators of a lattice and turn them into a matrix.
    \begin{lstlisting}
    sage: p = 59
    sage: B.<i,j,k> = QuaternionAlgebra(-1, -p)
    sage: gens = [1/2 + (2/2)*i + (3/2)*j + (4/2)*k, 5/2 + (6/2)*i + (7/2)*j + (8/2)*k]
    sage: from sage.algebras.quatalg.quaternion_algebra import quaternion_algebra_cython
    sage: Z, d = quaternion_algebra_cython.integral_matrix_and_denom_from_rational_quaternions(gens)
    sage: Z
    [1 2 3 4]
    [5 6 7 8]
    sage: d
    2
  \end{lstlisting}
\end{example}

We want to be able to ``row reduce'' a matrix representing a lattice.
To do this we need unimodular matrices.
Unimodular matrices are to lattices as elementary matrices are to vector spaces.

\begin{definition}[Unimodular matrix]
    A \emph{unimodular} matrix is a square integral matrix \( U \in \Z^{n \times n} \) with \( |\det U| = 1 \).
\end{definition}

\begin{lemma}[Properties of unimodular matrices]
    Let \( U \in \Z^{n \times n}\) be a unimodular matrix. Then
    \begin{enumerate}
        \item The inverse \( U^{-1} \) is also unimodular,
        \item \( x \in \Q^n \) is integral if and only if \( Ux \) is integral,
    \end{enumerate}
    If \( A, A' \) are matrices of full rank then
    \[
        A = A'U \text{ for some unimodular \( U \) if and only if } \Lambda(A) = \Lambda(A').
    \]
\end{lemma}
\begin{proof}
    (1) Since \( U \) is integral all its cofactors are integers.
    It follows that its matrix of minors is integral.
    Hence its inverse is integral since \( \det U = \pm 1 \) .
    Finally \( \det U^{-1} = 1 / (\det U) = \pm 1\).

    (2) If \( x \) is integral obviously \( Ux \) is integral.
    If \( Ux \) is integral then \( U^{-1}Ux = x \) is integral.

    For the second part suppose that \( A = A'U \) for some unimodular matrix \( U \).
    Then \( Ax = A'(Ux) \) and \( Ux \) is integral so \( \Lambda(A) \subset \Lambda(A') \).
    Simlarly \( A'x = A(U^{-1}x) \) so \( \Lambda(A) = \Lambda(A') \).

    Suppose that \( \Lambda(A) = \Lambda(A') \).
    Then the columns of \( A \) are in \( \Lambda(A') \) so \( A = A'V \) for some integral matrix \( V \).
    Similarly \( A' = AW \).
    Combining these two equations gives \( A = AWV \) so \( WV  = 1 \) which implies \( \det W \det V = 1 \) hence \( W \) (and \( V \)) is unimodular.
\end{proof}

\begin{definition}[Hermite normal form]
    Let \( A \) be a \( n \times m \) rational matrix.
    The \emph{Hermite normal form} of \( A \) is the unique matrix \( H \) such that there is a unimodular \( n \times n \) matrix \( U \) with \( UA = H \) such that \( H \) satisfies the following conditions
    \begin{enumerate}
        \item \( H \) is upper triangular,
        \item the pivot in each row is the largest element and the entries above it are nonnegative.
    \end{enumerate}
    The hermite normal form always exists.
\end{definition}

Note that \( \Lambda(A) = \Lambda(H) \) by adapting the argument in the above lemma.



\begin{example}
    Given a set of generators for a lattice we can find a set of four generators \( \alpha_1, \dots , \alpha_4 \) such that the matrix
    \[
        A =
        \begin{pmatrix}
            [\alpha_1] \\  \vdots \\ [\alpha_k]
        \end{pmatrix}.
    \]
    is in hermite normal form.
    The following code shows how to take a list of generators, turn them into a matrix, compute the hermite normal form and convert them back.
    \begin{lstlisting}
        sage: B.<i, j, k> = QuaternionAlgebra(-1, -59)
        sage: gens = [i+j, i-j, 2*k, B(1/3), i - k]
        sage:  Z, d = quaternion_algebra_cython.integral_matrix_and_denom_from_rational_quaternions(gens)
        sage: H = Z.hermite_form(include_zero_rows=False)
        sage: quaternion_algebra_cython.rational_quaternions_from_integral_matrix_and_denom(B, H, d)
  \end{lstlisting}
\end{example}

\begin{example}[Applications of HNF]
    You can solve a number of problems with hermite normal form. The results are \href{https://cseweb.ucsd.edu/classes/sp14/cse206A-a/}{here}.
    \begin{description}
        \item[Basis problem] Given a lattice as a matrix \( A \) compute a basis for the lattice (set of \( \Z \)-independent vectors which span the lattice).
        The nonzero rows of the HNF of \( A \) form a basis.
        \item [Equality] Given two lattices as matrices \( A_1, A_2 \) check if \( \Lambda(A_1) = \Lambda(A_2) \).
        To do this you check if \( HNF(A_1) = HNF(A_2) \).
        \item [Union] You find the hermite normal form of \( [A_1 | A_2] \).
        \item [Containment] You check if \( \Lambda(A_1) \cup \Lambda(A_2)  = \Lambda(A_1)\) using the union an equivalence problems above and if they do then \( \Lambda(A_2) \subset \Lambda(A_1) \).
        \item [Membership] Check if \( \Lambda([v]) \subset \Lambda(A) \).
        \item [Intersection] Done using duals.
    \end{description}
\end{example}

\begin{definition}[Order]
    Let \( B \) be quaternion algebra over \( \Q \).
    An \emph{order} \( O \subset B \) is a lattice that is a ring.
    In particular \( 1 \in O \).
\end{definition}

\begin{remark}
    To check if a lattice, represented by a matrix \( A \), is an order we first check if it has full rank, then if \( 1 \in O \) using the membership test from the example above and then check if the basis is multiplicatively closed.
\end{remark}

\begin{definition}[Order of a lattice]
    Let \( B \) be quaternion algebra over \( \Q \).
    Let \( I \subset B  \) be a lattice.
    The \emph{left order} of \( I \) is the set
    \[
        O_L(I) = \{\alpha \in B \mid \alpha I \subset I\}.
    \]
    The \emph{right order} of \( I \) is the set
    \[
        O_R(I) = \{ \alpha \in B \mid I \alpha \subset I \}.
    \]
    These are in fact always orders.
\end{definition}

\begin{remark}
    Let \( B \) be quaternion algebra over \( \Q \).
    Let \( I \subset B  \) be a lattice with basis \( \{ \beta_0, \beta_1, \beta_2, \beta_3 \} \).
    To find the left order of a lattice I we first observe that
    \begin{align*}
        O_L(I)
          & = \{ \alpha \in B \mid \alpha I \subset I \}                                    \\
          & = \{ \alpha \in B \mid \alpha \beta_j \subset I \text{ , for } j = 0, 1, 2, 3\} \\
          & = \bigcap_{j = 0, 1, 2, 3} \{ \alpha \in B \mid \alpha \beta_j  \in I \}        \\
          & = \bigcap_{j = 0, 1, 2, 3} \{ \alpha \in B \mid \alpha  \in I\beta_j^{-1} \}    \\
          & = \bigcap_{j = 0, 1, 2, 3} I\beta_j^{-1}                                        \\
    \end{align*}
    where
    \[
        I\beta_j^{-1}
        \vcentcolon = \{\alpha \beta_j^{-1} \mid \alpha \in I \}
        = \op{span}_\Z\{\beta_0\beta_j^{-1}, \beta_1\beta_j^{-1}, \beta_2\beta_j^{-1}, \beta_3\beta_j^{-1} \}.
    \]
    Note that \( I\beta_j^{-1} \) is a lattice and hence we can calculate \( \bigcap_{j = 0, 1, 2, 3} I\beta_j^{-1}  \) using lattice intersection algorithms.
    The following sage code does this.
    \begin{lstlisting}
        sage: M = [(~b).matrix() for b in self.basis()]
        sage: B = self.basis_matrix()
        sage: invs = [B*m for m in M]
        sage: ISB = [Q(v) for v in intersection_of_row_modules_over_ZZ(invs).row_module(ZZ).basis()]
        sage: O = A.quaternion_order(ISB)
  \end{lstlisting}
\end{remark}

\begin{definition}[Ideal]
    Let \( B \) be quaternion algebra over \( \Q \).
    Let \(O, O' \subset B\) be orders.
    A \textbf{left fractional $O$-ideal} is a lattice $I \subset B$ such that $O \subset O_L(I)$.
    A \textbf{right fractional $O$-ideal} is a lattice $I \subset B$ such that $O \subset O_R(I)$.
    A \textbf{fractional $O, O'$-ideal} is a lattice $I \subset B$ that is both a left fractional $O$-ideal and a right fractional $O'$-ideal.
\end{definition}

\begin{lemma}[Normal ideals are fractional ideals]
    Let \( B \) be quaternion algebra over \( \Q \).
    Let \(O \subset B\) be an order.
    Suppose \( I \subset O \) is an ideal in the usual sense and \( I \) is full (as a lattice).
    Then \( I \) is a left fractional \( O \)-ideal.
\end{lemma}

\begin{definition}[Integral ideals]
    Let \( B \) be quaternion algebra over \( \Q \).
    Let \(O \subset B\) be an order.
    Let \( I  \subset B \) be a left fractional \( O \)-ideal.
    An \emph{integral ideal} \( I  \subset B \) is a left fractional \( O \)-ideal such that \( I \subset O_L(I) \).
    Equivalently, a left fractional \( O \)-ideal \( I  \subset B \) is an integral ideal if \( I \subset O_R(I) \).
\end{definition}

\begin{remark}
    Note that if \( I \subset O \) then \( I \) is immediatly integral since \( O \subset O_L(I) \).
    Any ideal can be made integral by ``clearning denominators''.
    That is, there always exists some \( d \in \Z \) such that \( Id \) is integral.
\end{remark}

\begin{definition}[Norm of an ideal]
    Let \( B \) be quaternion algebra over \( \Q \).
    Let \(O \subset B\) be an order.
    Let \( I  \subset B \) be a left fractional \( O \)-ideal.
    The reduced norm of \( I \) is
    \[
        \nrd(I) = \op{span}_\Z\{ \nrd(\alpha) \mid \alpha \in I\}.
    \]
\end{definition}

\begin{remark}[Why the norm is reduced]
    Let \( B = (a, b | \Q )\).
    So \( i^2 = a \), \( j^2 = b \) and \( ij = k = -ji \).
    Recall that if \( \alpha = t + xi + yj + zk \) then the reduced norm of \( \alpha \) is
    \[
        \nrd(\alpha) = \alpha \overline{\alpha} = t^2 - ax^2 - by^2 + abz^2.
    \]
    There is another obvious way to define the norm and trace of $\alpha \in B$.
    We can embed $B = (a, b \mid F)$ into the endomorphism ring $\End_F(B)$ under the left regular representation
    \begin{align*}
          & B \hookrightarrow \End_F(B) \iso M_4(F)                       \\
          & \alpha \mapsto (\lambda_{\alpha}: \beta \mapsto \alpha \beta)
    \end{align*}
    where $B$ is considered as 4 dimensional $F$-vector space.
    Then we could simply define
    \begin{align*}
        \op{N}(\alpha) = \op{N}([\alpha])
    \end{align*}
    where $\op{N}([\alpha])$ is the norm of the matrix $[\alpha]\in M_4(F) $ corresponding to $\alpha \in B$.
    We can explicitly calculate the matrix \( [\alpha] \)with respect to the basis \( 1, i, j, k \).
    If \( \alpha = t + xi + yj + zk \) then
    \[
        [\alpha] =
        \begin{pmatrix}
            t    & x   & y   & z  \\
            xa   & t   & -za & -y \\
            y    & zb  & t   & x  \\
            -zab & -yb & xa  & t
        \end{pmatrix}
    \]
    Note you can do this in sage like this

    A natural choice for the norm of an element is the determinant of this matrix.
    When you do the calculation you get
    \[
        det([\alpha]) = \nrd(\alpha)^2.
    \]
    This is confirmed with the following calculation
    \begin{lstlisting}
        sage: R.<t, x, y, z, a, b> = PolynomialRing(ZZ)
        sage: A.<i, j, k> = QuaternionAlgebra(a, b)
        sage: alpha = t + x*i + y*j + z*k
        sage: alpha.matrix(action='left')
        [     t      x      y      z]
        [   x*a      t   -z*a     -y]
        [   y*b    z*b      t      x]
        [-z*a*b   -y*b    x*a      t]
        sage: det(alpha.matrix(action='left')) == alpha.reduced_norm()^2
        True
    \end{lstlisting}


\end{remark}

\begin{remark}[Calculating the norm of an ideal]
    {\color{red} I can only prove this works when \( O_L(I) \) is maximal but in Sage and Magma they seem to use this method across the board. I guess the idea idea }
    Calculating norms is far more tricky than what we have done so far.
    Let \( B = (a, b | \Q )\).
    So \( i^2 = a \), \( j^2 = b \) and \( ij = k = -ji \).
    Recall that if \( \alpha = t + xi + yj + zk \) then the reduced norm of \( \alpha \) is
    \[
        \nrd(\alpha) = \alpha \overline{\alpha} = t^2 - ax^2 - by^2 + abz^2.
    \]
    We will now show


    To calculate the norm of an ideal \( I \) with \( O_L(I) \) maximal with  we first define the notion of a \emph{discriminant}.
    Let \( B \) be quaternion algebra over \( \Q \).
    Let \( M \subset B \) be a lattice with basis \( \{\alpha_0, \alpha_1, \alpha_2, \alpha_3,\} \).
    The the discriminant of \( M \) is the determinant of the gram matrix
    \[
        \disc(M) = |\det(\trd(\alpha_i \alpha_j)_{i, j})|.
    \]
    From Lemma 15.4.8 of Voight we know that \( \disc(M) \) is always a square (note to use this lemma we need that \( M \) is projective but this is always true in our case from Corollary 29.28 of Voight).
    We can therefore define \( \op{discrd}(M) = \sqrt(\disc(M)). \)
    We now use the fact that if \( O_L(I) \) is maximal then \( I \) is invertible so by Main Theorem 16.1.3 in Voights book
    \[
        \nrd(I)^2 = [O_L(I) : I].
    \]
    We combine this with Lemma 15.3.8 in Voight which says that
    \[
        \disc(I) = [J : I]^2\disc(J)
    \]
    for any projective lattice \( I, J \subset B \) and we get
    \[
        \nrd(I)^2
        = [O_L(I) : I]
        = \sqrt{\disc(I) / \disc(O_L(I))}
        = \op{discrd}(I) / \op{discrd}(O_L(I)).
    \]
    Taking square roots gives the norm.
    This is how Sage does it.
    Magma computes \( [O_L(I) : I] \) by calculating the determinant of the change of basis matrix from \( I \) to \( O_L(I) \).
    See paragraph 9.6.3 for details.
\end{remark}

\begin{remark}[Ideals of \( M_2(\F_{p^n} \) ]
    Let \( R =  M_2(\F_{p^n}) \).
    Then the proper ideals of \( R \) are all of the form \( I = R\alpha \) where
    \[
        \alpha =
        \begin{pmatrix}
            a & b \\
            0 & 0
        \end{pmatrix}.
    \]
    Hence
    \[
        I = \left\{ \begin{pmatrix}
            c_1a & c_1b \\
            c_2a & c_2b
        \end{pmatrix} \mid c_i \in \F_{p^n} \right\}
    \]

    We now show that all ideals are of this form.
    Let \( I \subset R \) be a left \( R \)-ideal.
    The claim is that both of the rows of the elements \( \alpha \in I \) are of the form \( \begin{pmatrix} c_1a & c_1b\end{pmatrix} \) for some fixed \( a, b \in \F_{p^n} \).
    Suppose that
    \[
        \alpha =  \begin{pmatrix}
            a & b \\
            * & *
        \end{pmatrix}
    \]
    and
    \[
        \beta = \begin{pmatrix}
            c & d \\
            * & *
        \end{pmatrix}
    \]
    where \( (a, b) \) and \( (c, d) \) are linearly independent.
    Then
    \[
        \begin{pmatrix}
            a & b \\
            c & d
        \end{pmatrix} = \alpha + E \beta \in I
    \]
    where \( E \in M_2(\F_{p^n}) \) is the elementary matrix that swaps rows.
    Since \( (a, b) \) and \( (c, d) \) are linearly independent the matrix \( \begin{pmatrix}
        a & b \\
        c & d
    \end{pmatrix} \) is invertible and hence \( I \) contains a unit so \( I = R \).
    It follows that all proper ideals are of this form.

    Moreover \( R \) acts on its left ideals by right multiplication
    \[
        \begin{pmatrix}
            a & b \\
            0 & 0
        \end{pmatrix}
        \begin{pmatrix}
            c_1 & c_2 \\
            c_3 & c_4
        \end{pmatrix}
        =
        \begin{pmatrix}
            ac_1 + bc_3 & ac_2 + bc_4 \\
            0           & 0
        \end{pmatrix}
    \]
    and it's relatively clear this action is transitive for nontrivial proper ideals.

\end{remark}

\begin{lemma}[Ideal of \( M_2(\F_{p^n} \) ]
    There is a bijection
    \[
        \{\text{Ideals of } M_2(\F_{p^n}\} \leftrightarrow \{ \text{Subspace of } \F_{p^n}^2 \}.
    \]
    Given an ideal \( I \subset  M_2(\F_{p^n}\} \) the corresponding subspace is \( \cap_{\alpha \in I} \ker \alpha \) and given a subspace \( W \) the corresponding ideal is \( I = \{\alpha \in M_2(\F_{p^n} \mid W \subset \ker \alpha\} \).

    This implies that all the ideals \( 0 \neq I \subsetneq M_2(\F_{p^n} \) are principle as the correspond tho one dimensional subspaces.
\end{lemma}
\begin{proof}
    TODO.
\end{proof}

\begin{lemma}[\( I = ON + O\alpha \)]
    Let \( B \) be quaternion algebra over \( \Q \).
    Let \( O \subset B \) be a maximal order.
    Let \( I \subsetneq O \) be a left \( O \)-ideal with prime norm \( N = \nrd(I) \).
    Let \( \alpha \in I \) and suppose \( \alpha \not\in NO \).
    Then
    \[
        I = ON + O \alpha .
    \]
\end{lemma}
\begin{proof}
    {\color{red} This is an exercise in Voights book. The following proof is probably not the one he had in mind when he wrote the exercise.}
    Since \( N = \nrd(I) \) we know that \( ON \subset I \).
    Thus from the correspondence theorem \( I / NO \) is an ideal in \( O / NO \).
    There is a algebra isomorphism \( O / NO \iso M_2(\Z / N\Z) \).
    We will actually prove this result in Lemma \ref{lem: O / NO isomorphism} but we can steal the result from the future for now.
    From Lemma \ref{lem: ideals of M2} we know that all the left ideals of \( M_2(\Z / N\Z) \) are principal and generated by any nonzero element.
    Thus \(  \alpha + NO \in I / NO \) generates \( I / NO \) since \( \alpha \not\in NO \).

    Let \( \beta \in I \).
    Then since \( I \) is principle \( \beta + NO = (\gamma_1 + NO)(\alpha + NO) = \gamma_1\alpha + NO\) for some \( \gamma_1 + NO \in O / NO \).
    Thus \( \beta = \gamma_1\alpha + \gamma_2N \) for some \( \gamma_2 \in O \).
    Thus \( I \subset NO + O\alpha \).
    Since the other inclusion is obvious we have \( I =  NO + O\alpha \).
\end{proof}


\begin{lemma}[Computing connecting ideals]
    Let \( B \) be a rational quaternion algebra.
    Let \( O_1 = \op{span}_\Z \{\alpha_1, \alpha_2, \alpha_3, \alpha_4 \}, O_2 = \op{span}_\Z \{\beta_1, \beta_2, \beta_3, \beta_4 \} \) be maximal orders in \( B \).
    Let \( d \) be the lowest common multiple of the denominators of the entries in the matrix
    \[
        \begin{pmatrix}
            [\alpha_1] \\  \vdots \\ [\alpha_4]
        \end{pmatrix}
        \begin{pmatrix}
            [\beta_1] \\  \vdots \\ [\beta_4]
        \end{pmatrix}^{-1}.
    \]
    Then \( d O_1 \subset O_2 \) and
    \[
        I = \span_\Z \{d\alpha_i\beta_j \mid i, j \in \{1, \dots , 4\} \}
    \]
    is an \( O_1, O_2 \)-ideal.
\end{lemma}
\begin{proof}
    Note that \( \alpha_1, \dots , \alpha_4 \) is a basis for \( \Q^4 \) since \( O \) is a full as a lattice.
    So there exists \( c_{ij} \in \Q \) such that
    \[
        \beta_i = \sum_{j} c_{ij} \alpha_j
    \]
    for each \( i \in \{1, 2, 3, 4\} \).
    Let \( d = \lcm(\{c_{ij} \mid 1 \leq i , j \leq 4 \}) \) then
    \[
        d\beta_i \in O_1
    \]
    for all \( i \).
    Thus
    \[
        I = \op{span}_\Z \{d\alpha_j \beta_i \mid 1 \leq i, j \leq 4 \} \subset O_1.
    \]
    Finally it's clear that
    \[
        O_1 \subset O_L(I)
    \]
    and since \( O_1 \) is maximal \( I \) is a left \( O_1 \)-ideal.
    Similarly \( I \) is a  a right \( O_2 \)-ideal.
\end{proof}




\end{document}




































